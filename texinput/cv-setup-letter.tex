% Configure page margins with geometry
\geometry{left=2.5cm, top=2.5cm, right=2.5cm, bottom=2.5cm, footskip=.5cm}

% Specify the location of the included fonts
\fontdir[fonts/]

% Color for highlights
% Awesome Colors: awesome-emerald, awesome-skyblue, awesome-red, awesome-pink, awesome-orange
%                 awesome-nephritis, awesome-concrete, awesome-darknight
\colorlet{awesome}{awesome-orange}
% Uncomment if you would like to specify your own color
% \definecolor{awesome}{HTML}{CA63A8}

% Colors for text
% Uncomment if you would like to specify your own color
% \definecolor{darktext}{HTML}{414141}
% \definecolor{text}{HTML}{333333}
% \definecolor{graytext}{HTML}{5D5D5D}
% \definecolor{lighttext}{HTML}{999999}

% Set false if you don't want to highlight section with awesome color
\setbool{acvSectionColorHighlight}{true}

% If you would like to change the social information separator from a pipe (|) to something else
%\renewcommand{\acvHeaderSocialSep}{\quad\textbar\quad}
\renewcommand{\acvHeaderSocialSep}{\quad\cdotp\quad}
\renewcommand{\acvHeaderIconSep}{~}

%\newcommand{\cvcite}[1]{\citetitle{#1}~\cite{#1}}
\newcommand{\cvcite}[1]{\cite{#1}}

% defernumbers=true makes the "Publications" section label the entries
% consecutively, instead of in some semi-random order determined by
% LaTeX.
\usepackage[defernumbers=true,style=numeric,sorting=ydnt]{biblatex}
\addbibresource{zamboni-pubs.bib}
\addbibresource{zamboni-patents.bib}
\defbibheading{cvbibsection}[\bibname]{\cvsubsection{#1}}

% Some font redefinitions
\renewcommand*{\headerfirstnamestyle}[1]{{\fontsize{24pt}{1em}\headerfontlight\color{graytext} #1}}
\renewcommand*{\headerlastnamestyle}[1]{{\fontsize{24pt}{1em}\headerfont\bfseries\color{text} #1}}
\renewcommand*{\bodyfontlight}{\sourcesanspro}
\renewcommand*{\bibfont}{\paragraphstyle}
\renewcommand*{\entrylocationstyle}[1]{{\fontsize{10pt}{1em}\bodyfontlight\slshape\color{awesome} #1}}
\renewcommand*{\subsectionstyle}{\entrytitlestyle}

% For producing the "private" (non web site) version of the CV,
% set \makeprivatevita=1. By default, the "public" version is
% produced, for safety
\newif\ifprivatevita
\ifx\makeprivatevita\undefined
  \privatevitafalse
\else
  \privatevitatrue
\fi

% The \privatevita and \publicvita commands can be used
% to make things appear only in the corresponding version.
\newcommand{\privatevita}[1]{\ifprivatevita #1 \fi}
\newcommand{\publicvita}[1]{\ifprivatevita\else #1 \fi}

%-------------------------------------------------------------------------------
%	PERSONAL INFORMATION
%	Comment any of the lines below if they are not required
%-------------------------------------------------------------------------------
% Available options: circle|rectangle,edge/noedge,left/right
%\photo[right,noedge]{./images/foto_diego.png}
\name{Diego}{Zamboni}
%\position{IT Security Architect{\enskip\cdotp\enskip}Computer Scientist{\enskip\cdotp\enskip}Team and Project Leader}

\privatevita{
 % \address{Glärnischstrasse 15, 8803 Rüschlikon, Switzerland}
  \mobile{+41-79-756-3717}
%  \extrainfo{Swiss residence permit B}
}

\email{diego@zzamboni.org}
%\homepage{zzamboni.org}
\linkedin{zzamboni}
%\github{zzamboni}
% \gitlab{zzamboni}
%\stackoverflow{5562}{zzamboni}
%\twitter{@zzamboni}
%\skype{zzamboni}
% \reddit{reddit-id}

% \quote{``Be the change that you want to see in the world."}
